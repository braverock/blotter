% !TeX root = RJwrapper.tex
\title{Capitalized Title Here}
\author{by Author One, Author Two}

\maketitle

\abstract{%
An abstract of less than 150 words.
}

% Any extra LaTeX you need in the preamble

\hypertarget{introduction}{%
\subsection{Introduction}\label{introduction}}

Introductory section which may include references in parentheses
\citep{R}, or cite a reference such as \citet{R} in the text.

\hypertarget{section-title-in-sentence-case}{%
\subsection{Section title in sentence
case}\label{section-title-in-sentence-case}}

This section may contain a figure such as Figure \ref{figure:rlogo}.

\begin{figure}[htbp]
  \centering
  \includegraphics{Rlogo}
  \caption{The logo of R.}
  \label{figure:rlogo}
\end{figure}

\hypertarget{another-section}{%
\subsection{Another section}\label{another-section}}

There will likely be several sections, perhaps including code snippets,
such as:

\begin{Schunk}
\begin{Sinput}
x <- 1:10
x
\end{Sinput}
\begin{Soutput}
#>  [1]  1  2  3  4  5  6  7  8  9 10
\end{Soutput}
\end{Schunk}

\hypertarget{summary}{%
\subsection{Summary}\label{summary}}

This file is only a basic article template. For full details of
\emph{The R Journal} style and information on how to prepare your
article for submission, see the
\href{https://journal.r-project.org/share/author-guide.pdf}{Instructions
for Authors}.

\bibliography{RJreferences}


\address{%
Author One\\
Affiliation\\
line 1\\ line 2\\
}
\href{mailto:author1@work}{\nolinkurl{author1@work}}

\address{%
Author Two\\
Affiliation\\
line 1\\ line 2\\
}
\href{mailto:author2@work}{\nolinkurl{author2@work}}

